\documentclass[UTF8]{ctexart}
\usepackage{amsmath}
\usepackage{geometry}
\usepackage{amssymb}
\usepackage{hyperref}

\geometry{a4paper, left=2cm, right=2cm, top=2cm, bottom=2cm}
\setlength{\parskip}{0.5em}
\setlength{\parindent}{2em}

\title{ \LARGE \textbf{数学建模期末复习} }
\author{Liam}
\date{} 

\hypersetup{
    colorlinks=true,
    linkcolor=black,  % 设置链接颜色
    filecolor=black,  % 文件链接颜色
    urlcolor=blue,    % URL链接颜色
    citecolor=black   % 文献引用链接颜色
}

\makeatletter
\renewcommand{\maketitle}{
    \begin{center}
        {\huge \bfseries \@title \par}
        \vskip 0.5em
        {\Large \@author \par}
    \end{center}
}
\makeatother

\begin{document}

\maketitle
\setcounter{tocdepth}{2} 
\tableofcontents 

\section {排队论模型}
\subsection {简单的排队论模型的建模方法}
最简单的排队论模型是M/M/1模型,其中:
\begin{itemize}
    \item \textbf{M} 代表到达过程服从泊松分布(Markovian Arrival Process)。
    \item \textbf{M} 代表服务时间服从指数分布(Markovian Service Process)。
    \item \textbf{1} 代表只有一个服务台。
\end{itemize}

\subsection {排队系统中的平均队长(Lq)}
平均队长指系统中平均排队等待的客户数。对于M/M/1模型,平均队长的计算公式为:
\[
L_q = \frac{\lambda^2}{\mu (\mu - \lambda)}
\]

\subsection {平均顾客数(L)}
平均顾客数指系统中(包括排队和正在服务的)平均客户数。对于M/M/1模型,平均顾客数的计算公式为:
\[
L = \frac{\lambda}{\mu - \lambda}
\]

\subsection {系统中顾客逗留时间(W)}
系统中顾客逗留时间是指顾客从进入系统到离开系统的总时间,包括等待时间和服务时间。对于M/M/1模型,系统中顾客逗留时间的计算公式为:
\[
W = \frac{1}{\mu - \lambda}
\]

\subsection {队列中顾客等待时间(Wq)}
队列中顾客等待时间是指顾客在系统中等待服务的时间,不包括服务时间。对于M/M/1模型,队列中顾客等待时间的计算公式为:
\[
W_q = \frac{\lambda}{\mu (\mu - \lambda)}
\]

\subsection {例子}
假设一个银行柜台的客户到达率为 \(\lambda=2\) 人/分钟,服务率为 \(\mu=3\) 人/分钟,那么我们可以计算以下指标:

\begin{itemize}
    \item 平均队长(Lq):
    \[
    L_q = \frac{2^2}{3(3-2)} = \frac{4}{3} = 1.33
    \]

    \item 平均顾客数(L):
    \[
    L = \frac{2}{3-2} = 2
    \]

    \item 系统中顾客逗留时间(W):
    \[
    W = \frac{1}{3-2} = 1 \text{ 分钟}
    \]

    \item 队列中顾客等待时间(Wq):
    \[
    W_q = \frac{2}{3(3-2)} = \frac{2}{3} = 0.67 \text{ 分钟}
    \]
\end{itemize}

\newpage

\section {规划论模型}
\subsection {线性规划(Linear Programming)}
\subsubsection {基本概念和理论}
\begin{itemize}
    \item \textbf{目标函数}:一个线性函数,通常表示为 \( z = c_1 x_1 + c_2 x_2 + \cdots + c_n x_n \),其中 \( c_i \) 是常数, \( x_i \) 是决策变量。
    \item \textbf{约束条件}:一组线性不等式或等式,通常表示为 \( A \mathbf{x} \leq \mathbf{b} \) 或 \( A \mathbf{x} = \mathbf{b} \)。
    \item \textbf{非负约束}:决策变量通常需要非负,即 \( x_i \geq 0 \)。
\end{itemize}

\subsubsection {数学模型}
一个典型的线性规划问题可以表示为:
\[
\begin{aligned}
\text{最大化} & \quad z = c_1 x_1 + c_2 x_2 + \cdots + c_n x_n \\
\text{约束条件} & \quad 
\begin{cases}
a_{11} x_1 + a_{12} x_2 + \cdots + a_{1n} x_n \leq b_1 \\
a_{21} x_1 + a_{22} x_2 + \cdots + a_{2n} x_n \leq b_2 \\
\vdots \\
a_{m1} x_1 + a_{m2} x_2 + \cdots + a_{mn} x_n \leq b_m \\
x_1, x_2, \ldots, x_n \geq 0
\end{cases}
\end{aligned}
\]

\subsubsection {求解方法}
\paragraph{单纯形法步骤}
\begin{enumerate}
    \item \textbf{标准化问题}:
        \begin{itemize}
            \item 将目标函数转换为标准形式(如最大化形式)。
            \item 将不等式约束转换为等式约束(引入松弛变量)。
        \end{itemize}
    \item \textbf{构造初始单纯形表}:
        \begin{itemize}
            \item 列出初始基可行解对应的单纯形表。
        \end{itemize}
    \item \textbf{迭代计算}:
        \begin{itemize}
            \item \textbf{选择进入基变量}:选择目标函数中系数为正的变量进入基(通常选择使目标函数增长最快的变量)。
            \item \textbf{选择离开基变量}:通过计算每个约束的检验数,选择使目标函数值最小的变量离开基。
            \item 更新单纯形表,重复上述过程,直到目标函数的所有系数非正(达到最优解)。
        \end{itemize}
    \item \textbf{读取最优解}:
        \begin{itemize}
            \item 从最终的单纯形表中读取最优解及其对应的目标函数值。
        \end{itemize}
\end{enumerate}

\paragraph{内点法步骤}
\begin{enumerate}
    \item \textbf{构造初始点}:
        \begin{itemize}
            \item 选择一个在可行域内的初始点。
        \end{itemize}
    \item \textbf{迭代计算}:
        \begin{itemize}
            \item 根据KKT条件,构造内点法的优化路径。
            \item 更新初始点,沿优化路径前进。
        \end{itemize}
    \item \textbf{停止条件}:
        \begin{itemize}
            \item 当达到预设的精度要求或迭代次数限制时,停止迭代,输出最优解。
        \end{itemize}
\end{enumerate}

\subsubsection{线性规划的图解法}
\paragraph{基本方法}
\begin{enumerate}
    \item \textbf{绘制可行域}:根据约束条件,在坐标平面上绘制出所有约束的不等式,确定可行域。
    \item \textbf{确定目标函数的等值线}:将目标函数等值线绘制在同一坐标平面上,并移动等值线找到使目标函数达到最大或最小值的位置。
    \item \textbf{寻找最优解}:目标函数的最优解通常出现在可行域的顶点上,计算可行域所有顶点的目标函数值,选择最优解。
\end{enumerate}

\paragraph{例子}
考虑如下线性规划问题:
\[
\begin{aligned}
\text{最大化} & \quad z = 3x_1 + 5x_2 \\
\text{约束条件} & \quad 
\begin{cases}
2x_1 + 4x_2 \leq 40 \\
x_1 \leq 8 \\
x_2 \leq 10 \\
x_1, x_2 \geq 0
\end{cases}
\end{aligned}
\]
1. 绘制可行域:
    \begin{itemize}
        \item 约束 \( 2x_1 + 4x_2 \leq 40 \) 对应直线 \( x_1 + 2x_2 = 20 \)
        \item 约束 \( x_1 \leq 8 \)
        \item 约束 \( x_2 \leq 10 \)
        \item 绘制出这些直线并确定交点。
    \end{itemize}
2. 确定目标函数的等值线:
    \begin{itemize}
        \item 例如,绘制 \( 3x_1 + 5x_2 = z \) 的等值线并逐步移动。
    \end{itemize}
3. 寻找最优解:
    \begin{itemize}
        \item 计算可行域顶点(如 (0,0), (8,0), (0,10), (8,6), (5,7.5))的目标函数值。
        \item 找到目标函数值最大的点即为最优解。
    \end{itemize}

\subsection {整数规划(Integer Programming)}
\subsubsection {基本概念和理论}
\begin{itemize}
    \item \textbf{整数规划}:所有或部分决策变量必须是整数。
    \item \textbf{混合整数规划}:包含整数变量和连续变量。
\end{itemize}

\subsubsection {数学模型}
一个典型的整数规划问题可以表示为:
\[
\begin{aligned}
\text{最大化} & \quad z = c_1 x_1 + c_2 x_2 + \cdots + c_n x_n \\
\text{约束条件} & \quad 
\begin{cases}
a_{11} x_1 + a_{12} x_2 + \cdots + a_{1n} x_n \leq b_1 \\
a_{21} x_1 + a_{22} x_2 + \cdots + a_{2n} x_n \leq b_2 \\
\vdots \\
a_{m1} x_1 + a_{m2} x_2 + \cdots + a_{mn} x_n \leq b_m \\
x_1, x_2, \ldots, x_n \geq 0 \\
x_1, x_2, \ldots, x_k \in \mathbb{Z}
\end{cases}
\end{aligned}
\]

\subsubsection{整数规划的手算方法}
\paragraph{基本方法}
整数规划是在线性规划的基础上增加了整数约束,常用的方法有分支定界法和割平面法。
\paragraph{分支定界法}
1. 求解松弛问题:忽略整数约束,求解线性规划松弛问题。
2. 分支:选择一个非整数解的变量,将其分支为两个子问题。
3. 界定:对于每个子问题,计算上下界,剪枝不可能的分支。
4. 迭代:重复步骤2和3,直到找到最优整数解。

\paragraph{例子}
考虑如下整数规划问题:
\[
\begin{aligned}
\text{最大化} & \quad z = 3x_1 + 2x_2 \\
\text{约束条件} & \quad 
\begin{cases}
x_1 + x_2 \leq 4 \\
2x_1 + x_2 \leq 5 \\
x_1, x_2 \geq 0 \\
x_1, x_2 \in \mathbb{Z}
\end{cases}
\end{aligned}
\]
1. 求解松弛问题:忽略整数约束,得到松弛问题的解 \(x_1 = 2.5, x_2 = 1.5\),对应的目标函数值 \(z = 3 \cdot 2.5 + 2 \cdot 1.5 = 9\)。
2. 分支:选择非整数解变量 \(x_1\),创建两个子问题 \(x_1 \leq 2\) 和 \(x_1 \geq 3\)。
3. 求解子问题:
    - 子问题1:在 \(x_1 \leq 2\) 的条件下,解得 \(x_1 = 2, x_2 = 1\),目标函数值 \(z = 3 \cdot 2 + 2 \cdot 1 = 8\)。
    - 子问题2:在 \(x_1 \geq 3\) 的条件下,解得 \(x_1 = 2.5\),该解不可行。
4. 选择最优解:子问题1的解 \(x_1 = 2, x_2 = 1\) 是最优整数解。

\paragraph{割平面法}
1. 求解松弛问题:得到一个非整数解。
2. 构造割平面:添加新的约束,使得当前非整数解不可行。
3. 求解新的松弛问题:重复步骤1和2,直到找到整数解。

\paragraph{例子}
考虑同样的整数规划问题,通过割平面法求解:
1. 求解松弛问题:解得 \(x_1 = 2.5, x_2 = 1.5\)。
2. 构造割平面:添加约束 \(x_1 + x_2 \leq 3.5\)。
3. 求解新的松弛问题:解得 \(x_1 = 2, x_2 = 1\),是整数解。

\subsubsection {求解方法}
\paragraph{分支定界法步骤}
\begin{enumerate}
    \item \textbf{初始化}:
        \begin{itemize}
            \item 从线性规划的松弛问题开始,得到一个初始解。
        \end{itemize}
    \item \textbf{构建分支树}:
        \begin{itemize}
            \item 在当前解中选择一个非整数变量,构建两个分支(将该变量取上下限)。
        \end{itemize}
    \item \textbf{求解子问题}:
        \begin{itemize}
            \item 对每个分支求解线性规划子问题。
        \end{itemize}
    \item \textbf{剪枝}:
        \begin{itemize}
            \item 如果子问题的解不可行或目标函数值不优于已知最优解,则剪去该分支。
        \end{itemize}
    \item \textbf{迭代}:
        \begin{itemize}
            \item 重复上述步骤,直到所有分支均已求解或剪枝。
        \end{itemize}
    \item \textbf{输出最优解}:
        \begin{itemize}
            \item 从已找到的所有可行解中选择目标函数值最优的解。
        \end{itemize}
\end{enumerate}

\paragraph{割平面法步骤}
\begin{enumerate}
    \item \textbf{初始解}:
        \begin{itemize}
            \item 从线性规划的松弛问题开始,得到一个初始解。
        \end{itemize}
    \item \textbf{构造割平面}:
        \begin{itemize}
            \item 根据当前解,构造新的约束(割平面),以排除当前解但保留所有整数可行解。
        \end{itemize}
    \item \textbf{更新模型}:
        \begin{itemize}
            \item 将新约束加入模型,求解更新后的线性规划问题。
        \end{itemize}
    \item \textbf{迭代}:
        \begin{itemize}
            \item 重复上述步骤,直到找到整数解或无法构造新的割平面。
        \end{itemize}
    \item \textbf{输出最优解}:
        \begin{itemize}
            \item 当找到整数解时,输出最优解。
        \end{itemize}
\end{enumerate}

\subsection {目标规划(Goal Programming)}
\subsubsection {基本概念和理论}
\begin{itemize}
    \item \textbf{目标规划}:用于处理多个目标的问题,通过设定优先级或加权的方法,将多个目标整合为一个目标函数。
    \item \textbf{偏差变量}:表示目标与实际值的偏差,通过最小化这些偏差,达到各个目标。
\end{itemize}

\subsubsection {数学模型}
一个典型的目标规划问题可以表示为:
\[
\begin{aligned}
\text{最小化} & \quad P = \sum_{i=1}^k w_i (d_i^+ + d_i^-) \\
\text{约束条件} & \quad 
\begin{cases}
\sum_{j=1}^n a_{ij} x_j + d_i^- - d_i^+ = b_i \quad (i=1,2,\ldots,k) \\
x_j \geq 0, \quad d_i^+, d_i^- \geq 0
\end{cases}
\end{aligned}
\]

其中,\( w_i \) 是优先级权重,\( d_i^+ \) 和 \( d_i^- \) 分别表示正偏差和负偏差。

\subsubsection {求解方法}
\paragraph{加权和法步骤}
\begin{enumerate}
    \item \textbf{设定权重}:
        \begin{itemize}
            \item 根据各个目标的重要性,设定权重 \( w_i \)。
        \end{itemize}
    \item \textbf{构造目标函数}:
        \begin{itemize}
            \item 将各个目标的偏差变量加权求和,构造总的目标函数 \( P \)。
        \end{itemize}
    \item \textbf{求解线性规划问题}:
        \begin{itemize}
            \item 使用线性规划求解方法,求解构造的目标函数。
        \end{itemize}
    \item \textbf{输出最优解}:
        \begin{itemize}
            \item 读取最优解及其偏差变量值。
        \end{itemize}
\end{enumerate}

\paragraph{优先级法步骤}
\begin{enumerate}
    \item \textbf{确定优先级}:
        \begin{itemize}
            \item 根据各个目标的重要性,设定优先级。
        \end{itemize}
    \item \textbf{逐次优化}:
        \begin{itemize}
            \item 从最高优先级目标开始,逐次优化目标函数。
            \item 在每一步优化过程中,将较低优先级目标作为约束条件加入模型。
        \end{itemize}
    \item \textbf{迭代求解}:
        \begin{itemize}
            \item 重复上述步骤,直到所有目标均已优化。
        \end{itemize}
    \item \textbf{输出最优解}:
        \begin{itemize}
            \item 读取最终的最优解及各个目标的偏差变量值。
        \end{itemize}
\end{enumerate}

\subsection {动态规划(Dynamic Programming)}
\subsubsection {基本概念和理论}
\begin{itemize}
    \item \textbf{动态规划}:将问题分解为子问题,通过递归和记忆化存储(即避免重复计算),求解复杂问题。
    \item \textbf{贝尔曼方程}:描述最优值的递归关系,是动态规划的核心。
\end{itemize}

\subsubsection {数学模型}
一个典型的动态规划问题可以表示为:
\[
V_n(s_n) = \max_{a_n} \{ R_n(s_n, a_n) + \beta V_{n+1}(s_{n+1}) \}
\]

其中,\( V_n(s_n) \) 是第 \( n \) 阶段状态 \( s_n \) 的最优值,\( R_n(s_n, a_n) \) 是在状态 \( s_n \) 采取行动 \( a_n \) 的即时奖励,\( \beta \) 是折扣因子。

\subsubsection{动态规划的手算方法}
\paragraph{基本方法}
\begin{enumerate}
    \item 确定子问题:将原问题分解为若干子问题。
    \item 确定状态变量:定义每个子问题的状态。
    \item 确定状态转移方程:找出状态之间的递推关系。
    \item 确定初始条件和边界条件:初始化动态规划表。
    \item 填表计算:自底向上填表计算每个子问题的解。
\end{enumerate}

\paragraph{例子}
假设有一个背包问题:背包容量为5,物品有3件,其重量和价值分别为(2,3),(3,4),(4,5)。求最大价值。
1. 确定子问题:求容量为 \( j \) 的背包,前 \( i \) 件物品的最大价值 \( f(i, j) \)。
2. 状态转移方程:
    \[
    f(i, j) = \max(f(i-1, j), f(i-1, j-w_i) + v_i)
    \]
3. 初始条件和边界条件:
    \[
    f(0, j) = 0, \quad f(i, 0) = 0
    \]
4. 填表计算:
   \[
    \begin{array}{c|cccccc}
    i \backslash j & 0 & 1 & 2 & 3 & 4 & 5 \\
    \hline
    0 & 0 & 0 & 0 & 0 & 0 & 0 \\
    1 & 0 & 0 & 3 & 3 & 3 & 3 \\
    2 & 0 & 0 & 3 & 4 & 4 & 7 \\
    3 & 0 & 0 & 3 & 4 & 5 & 7 \\
    \end{array}
    \]
5. 最优解:最大价值为7。


\subsubsection {求解方法}
\paragraph{记忆化搜索步骤}
\begin{enumerate}
    \item \textbf{定义子问题}:
        \begin{itemize}
            \item 将原问题分解为若干子问题,定义每个子问题的最优值。
        \end{itemize}
    \item \textbf{递归求解}:
        \begin{itemize}
            \item 使用递归的方法求解子问题。
            \item 在每次递归调用时,检查子问题是否已计算过,如已计算则直接返回存储的结果。
        \end{itemize}
    \item \textbf{存储结果}:
        \begin{itemize}
            \item 将每个子问题的结果存储起来,避免重复计算。
        \end{itemize}
    \item \textbf{构造最优解}:
        \begin{itemize}
            \item 根据递归关系,逐步构造原问题的最优解。
        \end{itemize}
\end{enumerate}

\paragraph{递推法步骤}
\begin{enumerate}
    \item \textbf{初始化}:
        \begin{itemize}
            \item 定义最小子问题,并求解其最优值。
        \end{itemize}
    \item \textbf{自底向上求解}:
        \begin{itemize}
            \item 从最小子问题开始,逐步求解较大问题的最优值。
        \end{itemize}
    \item \textbf{构造最优解}:
        \begin{itemize}
            \item 使用递推关系,逐步构造原问题的最优解。
        \end{itemize}
    \item \textbf{输出最优解}:
        \begin{itemize}
            \item 从递推计算中读取最终的最优解。
        \end{itemize}
\end{enumerate}

\subsection {例子}
\subsubsection {线性规划例子}
某公司生产两种产品A和B,每种产品的利润分别为3和5,每种产品的生产时间分别为2小时和4小时。公司每天最多工作40小时。产品A和B每天最多可以生产8件和10件。如何安排生产以使利润最大?
\[
\begin{aligned}
\text{最大化} & \quad z = 3x_1 + 5x_2 \\
\text{约束条件} & \quad 
\begin{cases}
2x_1 + 4x_2 \leq 40 \\
x_1 \leq 8 \\
x_2 \leq 10 \\
x_1, x_2 \geq 0
\end{cases}
\end{aligned}
\]

\paragraph{求解步骤}
\begin{enumerate}
    \item \textbf{标准化问题}:
        \begin{itemize}
            \item 将目标函数转换为标准形式:\( z = 3x_1 + 5x_2 \)。
            \item 将不等式约束转换为等式约束,得到松弛变量:\( 2x_1 + 4x_2 + s_1 = 40 \),\( x_1 + s_2 = 8 \),\( x_2 + s_3 = 10 \),\( x_1, x_2, s_1, s_2, s_3 \geq 0 \)。
        \end{itemize}
    \item \textbf{构造初始单纯形表}:
        \begin{itemize}
            \item 初始基可行解:\( x_1 = 0 \),\( x_2 = 0 \),\( s_1 = 40 \),\( s_2 = 8 \),\( s_3 = 10 \)。
        \end{itemize}
    \item \textbf{迭代计算}:
        \begin{itemize}
            \item 选择进入基变量和离开基变量,更新单纯形表,直到目标函数的所有系数非正。
        \end{itemize}
    \item \textbf{读取最优解}:
        \begin{itemize}
            \item 从最终的单纯形表中读取最优解:\( x_1 = 8 \),\( x_2 = 6 \),最优目标函数值 \( z = 58 \)。
        \end{itemize}
\end{enumerate}

\subsubsection {整数规划例子}
某配送公司有3个仓库和5个客户,需要将货物从仓库运送到客户,求最优运输方案,使总运输成本最小。每个仓库的货物数量和每个客户的需求量均为整数。
\[
\begin{aligned}
\text{最小化} & \quad z = \sum_{i=1}^3 \sum_{j=1}^5 c_{ij} x_{ij} \\
\text{约束条件} & \quad 
\begin{cases}
\sum_{j=1}^5 x_{ij} \leq S_i \quad (i=1,2,3) \\
\sum_{i=1}^3 x_{ij} = D_j \quad (j=1,2,3,4,5) \\
x_{ij} \geq 0, \quad x_{ij} \in \mathbb{Z}
\end{cases}
\end{aligned}
\]

\paragraph{求解步骤}
\begin{enumerate}
    \item \textbf{初始化}:
        \begin{itemize}
            \item 从线性规划的松弛问题开始,得到一个初始解。
        \end{itemize}
    \item \textbf{构建分支树}:
        \begin{itemize}
            \item 在当前解中选择一个非整数变量,构建两个分支(将该变量取上下限)。
        \end{itemize}
    \item \textbf{求解子问题}:
        \begin{itemize}
            \item 对每个分支求解线性规划子问题。
        \end{itemize}
    \item \textbf{剪枝}:
        \begin{itemize}
            \item 如果子问题的解不可行或目标函数值不优于已知最优解,则剪去该分支。
        \end{itemize}
    \item \textbf{迭代}:
        \begin{itemize}
            \item 重复上述步骤,直到所有分支均已求解或剪枝。
        \end{itemize}
    \item \textbf{输出最优解}:
        \begin{itemize}
            \item 从已找到的所有可行解中选择目标函数值最优的解。
        \end{itemize}
\end{enumerate}

\subsubsection {目标规划例子}
某企业希望同时最小化生产成本和环境污染,但生产成本和环境污染存在冲突。企业可以通过设定两个目标,将问题转化为目标规划模型。
\[
\begin{aligned}
\text{最小化} & \quad P = w_1 (d_1^+ + d_1^-) + w_2 (d_2^+ + d_2^-) \\
\text{约束条件} & \quad 
\begin{cases}
f_1(x) + d_1^- - d_1^+ = \text{目标成本} \\
f_2(x) + d_2^- - d_2^+ = \text{目标污染} \\
x \geq 0, \quad d_1^+, d_1^-, d_2^+, d_2^- \geq 0
\end{cases}
\end{aligned}
\]

\paragraph{求解步骤}
\begin{enumerate}
    \item \textbf{设定权重}:
        \begin{itemize}
            \item 根据各个目标的重要性,设定权重 \( w_1 \) 和 \( w_2 \)。
        \end{itemize}
    \item \textbf{构造目标函数}:
        \begin{itemize}
            \item 将各个目标的偏差变量加权求和,构造总的目标函数 \( P \)。
        \end{itemize}
    \item \textbf{求解线性规划问题}:
        \begin{itemize}
            \item 使用线性规划求解方法,求解构造的目标函数。
        \end{itemize}
    \item \textbf{输出最优解}:
        \begin{itemize}
            \item 读取最优解及其偏差变量值。
        \end{itemize}
\end{enumerate}

\subsubsection {动态规划例子}
某人需要从A地到B地,途中需要经过若干中转站。每段路程的费用已知,求最小费用路径。
\[
V_i = \min_{j} \{ c_{ij} + V_j \}
\]

其中,\( V_i \) 是从节点 \( i \) 到终点的最小费用,\( c_{ij} \) 是从节点 \( i \) 到节点 \( j \) 的费用。

\paragraph{求解步骤}
\begin{enumerate}
    \item \textbf{定义子问题}:
        \begin{itemize}
            \item 将原问题分解为若干子问题,定义每个子问题的最优值 \( V_i \)。
        \end{itemize}
    \item \textbf{递归求解}:
        \begin{itemize}
            \item 使用递归的方法求解子问题。
            \item 在每次递归调用时,检查子问题是否已计算过,如已计算则直接返回存储的结果。
        \end{itemize}
    \item \textbf{存储结果}:
        \begin{itemize}
            \item 将每个子问题的结果存储起来,避免重复计算。
        \end{itemize}
    \item \textbf{构造最优解}:
        \begin{itemize}
            \item 根据递归关系,逐步构造原问题的最优解。
        \end{itemize}
    \item \textbf{输出最优解}:
        \begin{itemize}
            \item 从递推计算中读取最终的最优解。
        \end{itemize}
\end{enumerate}

\newpage

\section {插值与拟合建模}
\subsection {插值(Interpolation)}
\subsubsection {基本概念和理论}
插值是通过已知数据点构造一个函数,使得该函数经过所有已知数据点。常用的插值方法有多项式插值、分段线性插值和样条插值等。

\subsubsection {常用插值方法}
\paragraph{多项式插值}
\textbf{拉格朗日插值法}:通过构造拉格朗日插值多项式实现插值。
拉格朗日插值多项式的公式为:
\[
P(x) = \sum_{i=0}^{n} y_i \ell_i(x)
\]
其中,\(\ell_i(x)\) 是拉格朗日基函数,定义为:
\[
\ell_i(x) = \prod_{\substack{0 \leq j \leq n \\ j \neq i}} \frac{x - x_j}{x_i - x_j}
\]

\paragraph{求解步骤}
\begin{enumerate}
    \item 确定插值点 \( (x_0, y_0), (x_1, y_1), \ldots, (x_n, y_n) \)。
    \item 计算每个基函数 \(\ell_i(x)\)。
    \item 将基函数代入拉格朗日插值多项式公式,求得 \(P(x)\)。
\end{enumerate}

\textbf{牛顿插值法}:通过构造牛顿插值多项式实现插值。
牛顿插值多项式的公式为:
\[
P(x) = a_0 + a_1 (x - x_0) + a_2 (x - x_0)(x - x_1) + \cdots + a_n (x - x_0)(x - x_1) \cdots (x - x_{n-1})
\]
其中,系数 \(a_i\) 通过差商表计算得到。

\paragraph{求解步骤}
\begin{enumerate}
    \item 确定插值点 \( (x_0, y_0), (x_1, y_1), \ldots, (x_n, y_n) \)。
    \item 构造差商表,计算差商。
    \item 将差商代入牛顿插值多项式公式,求得 \(P(x)\)。
\end{enumerate}

\paragraph{分段线性插值}
分段线性插值是将相邻的已知数据点用线段连接起来,构造一个分段函数。
\paragraph{求解步骤}
\begin{enumerate}
    \item 确定插值点 \( (x_0, y_0), (x_1, y_1), \ldots, (x_n, y_n) \)。
    \item 对于每一对相邻的点 \( (x_i, y_i) \) 和 \( (x_{i+1}, y_{i+1}) \),构造线性插值函数:
    \[
    L_i(x) = y_i + \frac{y_{i+1} - y_i}{x_{i+1} - x_i} (x - x_i)
    \]
    \item 将所有线性插值函数组合成分段线性插值函数。
\end{enumerate}

\paragraph{样条插值}
样条插值是通过构造样条函数(通常是三次样条函数)实现插值。
\paragraph{求解步骤}
\begin{enumerate}
    \item 确定插值点 \( (x_0, y_0), (x_1, y_1), \ldots, (x_n, y_n) \)。
    \item 对于每一对相邻的点 \( (x_i, y_i) \) 和 \( (x_{i+1}, y_{i+1}) \),构造三次样条函数:
    \[
    S_i(x) = a_i + b_i (x - x_i) + c_i (x - x_i)^2 + d_i (x - x_i)^3
    \]
    \item 根据插值点的条件和连续性条件,建立方程组,求解系数 \(a_i, b_i, c_i, d_i\)。
    \item 将所有三次样条函数组合成整体的样条插值函数。
\end{enumerate}

\subsection {拟合(Fitting)}
\subsubsection {基本概念和理论}
拟合是通过一组数据点构造一个函数,使得该函数尽量逼近这些点。常用的拟合方法有最小二乘法、非线性拟合等。

\subsubsection {常用拟合方法}
\paragraph{最小二乘法}
最小二乘法通过最小化误差平方和,找到最优拟合函数。
线性最小二乘法:
\begin{enumerate}
    \item \textbf{确定模型}:假设拟合函数为线性模型 \( y = ax + b \)。
    \item \textbf{构造误差平方和}:构造误差平方和函数:
    \[
    S = \sum_{i=1}^{n} (y_i - (ax_i + b))^2
    \]
    \item \textbf{求解参数}:对误差平方和函数 \( S \) 求导,得到关于 \( a \) 和 \( b \) 的方程组:
    \[
    \frac{\partial S}{\partial a} = 0, \quad \frac{\partial S}{\partial b} = 0
    \]
    解方程组,得到最优参数 \( a \) 和 \( b \)。
\end{enumerate}

非线性最小二乘法:
\begin{enumerate}
    \item \textbf{确定模型}:假设拟合函数为非线性模型 \( y = f(x, \mathbf{a}) \),其中 \( \mathbf{a} \) 是待求参数向量。
    \item \textbf{构造误差平方和}:构造误差平方和函数:
    \[
    S = \sum_{i=1}^{n} (y_i - f(x_i, \mathbf{a}))^2
    \]
    \item \textbf{迭代求解}:使用迭代算法(如梯度下降法、牛顿法等),最小化误差平方和函数 \( S \),得到最优参数 \( \mathbf{a} \)。
\end{enumerate}

\paragraph{多项式拟合}
多项式拟合是通过拟合多项式函数逼近数据点。
\paragraph{求解步骤}
\begin{enumerate}
    \item \textbf{确定模型}:假设拟合函数为多项式模型 \( y = a_0 + a_1 x + a_2 x^2 + \cdots + a_m x^m \)。
    \item \textbf{构造误差平方和}:构造误差平方和函数:
    \[
    S = \sum_{i=1}^{n} (y_i - (a_0 + a_1 x_i + a_2 x_i^2 + \cdots + a_m x_i^m))^2
    \]
    \item \textbf{求解参数}:对误差平方和函数 \( S \) 求导,得到关于 \( a_0, a_1, \ldots, a_m \) 的方程组:
    \[
    \frac{\partial S}{\partial a_j} = 0 \quad (j=0, 1, \ldots, m)
    \]
    解方程组,得到最优参数 \( a_0, a_1, \ldots, a_m \)。
\end{enumerate}

\subsection {用插值拟合作数据处理}
\paragraph{插值拟合的应用步骤}
\begin{enumerate}
    \item \textbf{收集数据}:收集已知数据点 \( (x_0, y_0), (x_1, y_1), \ldots, (x_n, y_n) \)。
    \item \textbf{选择方法}:根据数据特点和需求,选择适当的插值或拟合方法。
    \item \textbf{构造模型}:根据选择的方法,构造插值或拟合函数。
    \item \textbf{求解模型}:使用相应的数学方法,求解插值或拟合函数的参数。
    \item \textbf{验证模型}:使用已知数据点或新数据点,验证插值或拟合函数的精度。
    \item \textbf{应用模型}:使用插值或拟合函数,对新数据进行预测或估计。
\end{enumerate}

\paragraph{插值拟合的例子}
假设我们有以下数据点:
\[
\begin{aligned}
(1, 2), \quad (2, 3), \quad (3, 5), \quad (4, 4)
\end{aligned}
\]
\paragraph{拉格朗日插值法}
\begin{enumerate}
    \item 确定插值点:
    \[
    (1, 2), \quad (2, 3), \quad (3, 5), \quad (4, 4)
    \]
    \item 计算拉格朗日基函数:
    \[
    \ell_0(x) = \frac{(x-2)(x-3)(x-4)}{(1-2)(1-3)(1-4)}, \quad \ell_1(x) = \frac{(x-1)(x-3)(x-4)}{(2-1)(2-3)(2-4)}, \quad \text{等等}
    \]
    \item 构造拉格朗日插值多项式:
    \[
    P(x) = 2\ell_0(x) + 3\ell_1(x) + 5\ell_2(x) + 4\ell_3(x)
    \]
\end{enumerate}

\paragraph{最小二乘法}
\begin{enumerate}
    \item 假设拟合函数为线性模型:
    \[
    y = ax + b
    \]
    \item 构造误差平方和函数:
    \[
    S = (2 - (a \cdot 1 + b))^2 + (3 - (a \cdot 2 + b))^2 + (5 - (a \cdot 3 + b))^2 + (4 - (a \cdot 4 + b))^2
    \]
    \item 对 \(a\) 和 \(b\) 求导并解方程组:
    \[
    \frac{\partial S}{\partial a} = 0, \quad \frac{\partial S}{\partial b} = 0
    \]
    解得 \( a \) 和 \( b \) 的最优值。
\end{enumerate}

\newpage

\section {回归分析模型}
\subsection {线性回归(Linear Regression)}
\subsubsection {基本概念和理论}
\begin{itemize}
    \item \textbf{线性回归模型}:假设因变量 \( y \) 与自变量 \( x \) 之间存在线性关系,模型表示为:
    \[
    y = \beta_0 + \beta_1 x + \epsilon
    \]
    其中,\(\beta_0\) 和 \(\beta_1\) 分别是截距和斜率,\(\epsilon\) 是误差项。
    \item \textbf{多元线性回归}:扩展到多元情况,即因变量 \( y \) 与多个自变量 \( x_1, x_2, \ldots, x_p \) 存在线性关系,模型表示为:
    \[
    y = \beta_0 + \beta_1 x_1 + \beta_2 x_2 + \cdots + \beta_p x_p + \epsilon
    \]
    \item \textbf{最小二乘法}:通过最小化误差平方和来估计模型参数 \(\beta_0\) 和 \(\beta_1\)。
\end{itemize}

\subsubsection {建模方法}
\begin{enumerate}
    \item \textbf{确定模型}:假设模型为 \( y = \beta_0 + \beta_1 x + \epsilon \)。
    \item \textbf{收集数据}:收集 \( n \) 个观测值 \((x_1, y_1), (x_2, y_2), \ldots, (x_n, y_n) \)。
    \item \textbf{构造误差平方和函数}:
    \[
    S = \sum_{i=1}^{n} (y_i - (\beta_0 + \beta_1 x_i))^2
    \]
    \item \textbf{求解参数}:对误差平方和函数 \( S \) 求导,得到关于 \(\beta_0\) 和 \(\beta_1\) 的方程组:
    \[
    \frac{\partial S}{\partial \beta_0} = 0, \quad \frac{\partial S}{\partial \beta_1} = 0
    \]
    解方程组,得到最优参数 \(\beta_0\) 和 \(\beta_1\)。
\end{enumerate}

\subsubsection {数理统计检验}
\paragraph{参数显著性检验}
\begin{itemize}
    \item \(t\) 检验用于检验 \(\beta_0\) 和 \(\beta_1\) 是否显著。
    \item 构造检验统计量:
    \[
    t = \frac{\hat{\beta}_i}{\text{SE}(\hat{\beta}_i)}
    \]
    其中,\(\hat{\beta}_i\) 是参数估计值,\(\text{SE}(\hat{\beta}_i)\) 是标准误差。
    \item 比较 \(t\) 值与临界值,判断是否拒绝原假设。
\end{itemize}

\paragraph{模型显著性检验}
\begin{itemize}
    \item \(F\) 检验用于检验整个回归模型的显著性。
    \item 构造检验统计量:
    \[
    F = \frac{\text{SSR}/p}{\text{SSE}/(n-p-1)}
    \]
    其中,\(\text{SSR}\) 是回归平方和,\(\text{SSE}\) 是误差平方和,\(p\) 是自变量个数。
    \item 比较 \(F\) 值与临界值,判断是否拒绝原假设。
\end{itemize}

\paragraph{拟合优度检验}
\begin{itemize}
    \item 判定系数 \(R^2\) 用于衡量模型的拟合优度:
    \[
    R^2 = 1 - \frac{\text{SSE}}{\text{SST}}
    \]
    其中,\(\text{SST}\) 是总平方和。
\end{itemize}

\subsection {非线性回归(Nonlinear Regression)}
\subsubsection {基本概念和理论}
\begin{itemize}
    \item \textbf{非线性回归模型}:假设因变量 \( y \) 与自变量 \( x \) 之间存在非线性关系,模型表示为:
    \[
    y = f(x, \beta) + \epsilon
    \]
    其中,\(f(x, \beta)\) 是非线性函数,\(\beta\) 是参数向量,\(\epsilon\) 是误差项。
    \item \textbf{最小二乘法}:通过最小化误差平方和来估计模型参数 \(\beta\)。
\end{itemize}

\subsubsection {建模方法}
\begin{enumerate}
    \item \textbf{确定模型}:假设模型为 \( y = f(x, \beta) + \epsilon \),如指数模型 \( y = \beta_0 e^{\beta_1 x} + \epsilon \)。
    \item \textbf{收集数据}:收集 \( n \) 个观测值 \((x_1, y_1), (x_2, y_2), \ldots, (x_n, y_n) \)。
    \item \textbf{构造误差平方和函数}:
    \[
    S = \sum_{i=1}^{n} (y_i - f(x_i, \beta))^2
    \]
    \item \textbf{迭代求解参数}:使用迭代算法(如梯度下降法、牛顿法等),最小化误差平方和函数 \( S \),得到最优参数 \(\beta\)。
\end{enumerate}

\subsubsection {数理统计检验}
\paragraph{参数显著性检验}
\begin{itemize}
    \item 与线性回归相似,通过 \(t\) 检验来检验参数是否显著。
\end{itemize}

\paragraph{模型显著性检验}
\begin{itemize}
    \item 与线性回归相似,通过 \(F\) 检验来检验模型的显著性。
\end{itemize}

\paragraph{拟合优度检验}
\begin{itemize}
    \item 与线性回归相似,通过判定系数 \(R^2\) 来衡量模型的拟合优度。
\end{itemize}

\subsection {用最小二乘方法建立回归分析模型}
\subsubsection {线性回归例子}
假设我们有以下数据点:
\[
\begin{aligned}
(x_1, y_1) = (1, 2), \quad (x_2, y_2) = (2, 3), \quad (x_3, y_3) = (3, 5), \quad (x_4, y_4) = (4, 4)
\end{aligned}
\]
\begin{enumerate}
    \item \textbf{确定模型}:假设模型为 \( y = \beta_0 + \beta_1 x + \epsilon \)。
    \item \textbf{构造误差平方和函数}:
    \[
    S = \sum_{i=1}^{4} (y_i - (\beta_0 + \beta_1 x_i))^2
    \]
    \item \textbf{求解参数}:对误差平方和函数 \( S \) 求导,得到关于 \(\beta_0\) 和 \(\beta_1\) 的方程组:
    \[
    \frac{\partial S}{\partial \beta_0} = 0, \quad \frac{\partial S}{\partial \beta_1} = 0
    \]
    解方程组,得到最优参数 \(\beta_0\) 和 \(\beta_1\)。
    \item \textbf{数理统计检验}:
        \begin{itemize}
            \item 进行 \(t\) 检验、\(F\) 检验和 \(R^2\) 检验,验证模型的显著性和拟合优度。
        \end{itemize}
\end{enumerate}

\subsubsection {非线性回归例子}
假设我们有以下数据点,并假设模型为指数模型 \( y = \beta_0 e^{\beta_1 x} + \epsilon \):
\[
\begin{aligned}
(x_1, y_1) = (1, 2.7), \quad (x_2, y_2) = (2, 7.4), \quad (x_3, y_3) = (3, 20.1), \quad (x_4, y_4) = (4, 54.6)
\end{aligned}
\]
\begin{enumerate}
    \item \textbf{确定模型}:假设模型为 \( y = \beta_0 e^{\beta_1 x} + \epsilon \)。
    \item \textbf{构造误差平方和函数}:
    \[
    S = \sum_{i=1}^{4} (y_i - \beta_0 e^{\beta_1 x_i})^2
    \]
    \item \textbf{迭代求解参数}:使用迭代算法(如梯度下降法),最小化误差平方和函数 \( S \),得到最优参数 \(\beta_0\) 和 \(\beta_1\)。
    \item \textbf{数理统计检验}:
        \begin{itemize}
            \item 进行 \(t\) 检验、\(F\) 检验和 \(R^2\) 检验,验证模型的显著性和拟合优度。
        \end{itemize}
\end{enumerate}

\newpage

\section {差分方程模型}
\subsection {差分法的基本思想}
\subsubsection {基本概念和理论}
\begin{itemize}
    \item \textbf{差分方程}:描述变量在离散时间点上的变化关系,类似于微分方程在连续时间点上的描述。
    \item \textbf{一阶差分方程}:最简单的差分方程,形式为:
    \[
    x_{n+1} = f(x_n)
    \]
    其中,\( x_n \) 是第 \( n \) 时刻的变量值,\( f \) 是某种函数。
    \item \textbf{高阶差分方程}:包含多个离散时间点的关系,形式为:
    \[
    x_{n+k} = f(x_n, x_{n+1}, \ldots, x_{n+k-1})
    \]
\end{itemize}

\subsubsection {差分法}
\begin{itemize}
    \item \textbf{差分法}:通过差分运算,将连续变量离散化,从而得到差分方程。常见的差分有前向差分、后向差分和中心差分。
    \begin{itemize}
        \item 前向差分:\( \Delta x_n = x_{n+1} - x_n \)
        \item 后向差分:\( \nabla x_n = x_n - x_{n-1} \)
        \item 中心差分:\( \delta x_n = \frac{x_{n+1} - x_{n-1}}{2} \)
    \end{itemize}
\end{itemize}

\subsection {建立实际问题的离散模型}
\subsubsection {离散模型建模步骤}
\begin{enumerate}
    \item \textbf{问题描述}:明确实际问题及其动态过程。
    \item \textbf{确定变量}:确定模型中的变量及其关系。
    \item \textbf{建立差分方程}:根据问题中的关系,建立相应的差分方程。
    \item \textbf{初始条件}:确定模型的初始条件,以便求解差分方程。
\end{enumerate}

\subsubsection {例子:人口增长模型}
假设一个地区的人口每年按固定增长率 \( r \) 增长,初始人口为 \( P_0 \)。
\begin{enumerate}
    \item \textbf{确定变量}:设 \( P_n \) 为第 \( n \) 年的人口。
    \item \textbf{建立差分方程}:根据固定增长率,得到差分方程:
    \[
    P_{n+1} = P_n (1 + r)
    \]
    \item \textbf{初始条件}:\( P_0 \) 为初始人口。
\end{enumerate}

\subsection {递推迭代法等求解过程}
\subsubsection {递推迭代法}
递推迭代法通过初始条件和差分方程,逐步计算后续变量值。
\begin{enumerate}
    \item \textbf{确定初始条件}:设 \( x_0 = x_0 \)。
    \item \textbf{递推公式}:根据差分方程,计算后续变量值:
    \[
    x_{n+1} = f(x_n)
    \]
    \item \textbf{迭代计算}:从初始条件出发,依次计算 \( x_1, x_2, \ldots, x_n \)。
\end{enumerate}

\subsubsection {例子:银行贷款问题}
假设某人从银行贷款 \( L \) 元,年利率为 \( r \),每年偿还 \( A \) 元。
\begin{enumerate}
    \item \textbf{确定变量}:设 \( B_n \) 为第 \( n \) 年末的贷款余额。
    \item \textbf{建立差分方程}:根据贷款利息和偿还金额,得到差分方程:
    \[
    B_{n+1} = B_n (1 + r) - A
    \]
    \item \textbf{初始条件}:\( B_0 = L \)。
    \item \textbf{递推迭代}:从初始条件出发,依次计算 \( B_1, B_2, \ldots, B_n \),直到贷款余额为零或负数。
\end{enumerate}

\subsection {蛛网模型}
\subsubsection {基本概念和理论}
\begin{itemize}
    \item \textbf{蛛网模型}:用于描述供需关系在市场中的动态调整,特别是价格与产量之间的关系。
    \item \textbf{基本假设}:生产者根据上期价格决定本期产量,消费者根据本期价格决定本期需求。
\end{itemize}

\subsubsection {建模步骤}
\begin{enumerate}
    \item \textbf{确定变量}:设 \( P_n \) 为第 \( n \) 期的价格,\( Q_n \) 为第 \( n \) 期的产量。
    \item \textbf{供给函数}:根据价格决定产量,设 \( Q_n = S(P_{n-1}) \)。
    \item \textbf{需求函数}:根据价格决定需求,设 \( Q_n = D(P_n) \)。
    \item \textbf{平衡条件}:供需平衡,得到差分方程:
    \[
    S(P_{n-1}) = D(P_n)
    \]
\end{enumerate}

\subsubsection {求解步骤}
\begin{enumerate}
    \item \textbf{确定供给函数和需求函数}:例如,线性供给函数 \( S(P_{n-1}) = a + b P_{n-1} \),线性需求函数 \( D(P_n) = c - d P_n \)。
    \item \textbf{建立差分方程}:
    \[
    a + b P_{n-1} = c - d P_n
    \]
    \item \textbf{递推迭代}:根据初始条件 \( P_0 \),迭代计算 \( P_1, P_2, \ldots, P_n \)。
\end{enumerate}

\subsection{差分方程的手算方法}
\subsubsection{基本方法}
\paragraph{差分方程的定义}
差分方程是描述变量在离散时间点上的变化关系,类似于微分方程在连续时间点上的描述。最常见的形式是一阶差分方程:
\[
x_{n+1} = f(x_n)
\]

\paragraph{手算步骤}
1. 确定初值:确定差分方程的初始值 \(x_0\)。
2. 递推计算:利用差分方程递推计算后续的值 \(x_1, x_2, \ldots\)。
3. 验证结果:检查计算结果是否符合差分方程和初始条件。

\paragraph{例子}
考虑差分方程 \(x_{n+1} = 2x_n + 1\),初始值 \(x_0 = 1\):
\[
\begin{aligned}
x_1 & = 2x_0 + 1 = 2 \cdot 1 + 1 = 3 \\
x_2 & = 2x_1 + 1 = 2 \cdot 3 + 1 = 7 \\
x_3 & = 2x_2 + 1 = 2 \cdot 7 + 1 = 15 \\
x_4 & = 2x_3 + 1 = 2 \cdot 15 + 1 = 31
\end{aligned}
\]


\subsection {例子总结}
\subsubsection {例子1:人口增长模型}
假设一个地区的人口每年按固定增长率 \( r \) 增长,初始人口为 \( P_0 \) = 1000,年增长率为 5\%。
\begin{enumerate}
    \item \textbf{确定变量}:设 \( P_n \) 为第 \( n \) 年的人口。
    \item \textbf{建立差分方程}:
    \[
    P_{n+1} = P_n (1 + 0.05)
    \]
    \item \textbf{初始条件}:\( P_0 = 1000 \)。
    \item \textbf{递推迭代}:计算 \( P_1, P_2, \ldots, P_n \):
    \[
    P_1 = 1000 \times 1.05 = 1050, \quad P_2 = 1050 \times 1.05 = 1102.5, \ldots
    \]
\end{enumerate}

\subsubsection {例子2:银行贷款问题}
假设某人从银行贷款 10000 元,年利率为 6\%,每年偿还 2000 元。
\begin{enumerate}
    \item \textbf{确定变量}:设 \( B_n \) 为第 \( n \) 年末的贷款余额。
    \item \textbf{建立差分方程}:
    \[
    B_{n+1} = B_n (1 + 0.06) - 2000
    \]
    \item \textbf{初始条件}:\( B_0 = 10000 \)。
    \item \textbf{递推迭代}:计算 \( B_1, B_2, \ldots, B_n \):
    \[
    B_1 = 10000 \times 1.06 - 2000 = 8600, \quad B_2 = 8600 \times 1.06 - 2000 = 7196, \ldots
    \]
\end{enumerate}

\subsubsection {例子3:蛛网模型}
假设市场供给函数 \( S(P_{n-1}) = 10 + 2P_{n-1} \),需求函数 \( D(P_n) = 40 - P_n \),初始价格 \( P_0 = 5 \)。
\begin{enumerate}
    \item \textbf{确定变量}:设 \( P_n \) 为第 \( n \) 期的价格。
    \item \textbf{建立差分方程}:根据供需平衡,\( 10 + 2P_{n-1} = 40 - P_n \)。
    \item \textbf{递推迭代}:根据初始条件 \( P_0 = 5 \),计算 \( P_1, P_2, \ldots, P_n \):
    \[
    10 + 2 \cdot 5 = 40 - P_1 \implies P_1 = 25, \quad 10 + 2 \cdot 25 = 40 - P_2 \implies P_2 = -20, \ldots
    \]
\end{enumerate}

\newpage

\section {微分方程模型}
\subsection {微分方程建模的基本步骤}
\subsubsection {建模步骤}
\begin{enumerate}
    \item \textbf{问题描述}:明确实际问题及其动态过程。
    \item \textbf{确定变量}:确定模型中的变量及其关系。
    \item \textbf{建立微分方程}:根据问题中的关系,建立相应的微分方程。
    \item \textbf{初始条件和边界条件}:确定模型的初始条件和边界条件,以便求解微分方程。
    \item \textbf{求解微分方程}:使用适当的数学方法求解微分方程。
    \item \textbf{验证模型}:验证模型的准确性和有效性。
    \item \textbf{应用模型}:将模型应用于实际问题,进行预测或分析。
\end{enumerate}

\subsection {线性微分方程建模基本方法}
\subsubsection {基本概念和理论}
\begin{itemize}
    \item \textbf{线性微分方程}:方程中变量及其导数是线性的。常见形式为:
    \[
    \frac{d^n y}{dx^n} + a_{n-1}(x) \frac{d^{n-1} y}{dx^{n-1}} + \cdots + a_1(x) \frac{dy}{dx} + a_0(x) y = f(x)
    \]
    其中,\( a_i(x) \) 和 \( f(x) \) 是已知函数,\( y \) 是待求解的函数。
\end{itemize}

\subsubsection {建模方法}
\begin{enumerate}
    \item \textbf{确定变量}:设 \( y(x) \) 为待求解函数。
    \item \textbf{建立微分方程}:根据实际问题,确定方程形式。例如,人口增长模型:
    \[
    \frac{dy}{dt} = ry
    \]
    \item \textbf{初始条件}:确定 \( y \) 在 \( x = x_0 \) 时的值 \( y(x_0) = y_0 \)。
    \item \textbf{求解微分方程}:使用适当的方法求解。
\end{enumerate}

\subsubsection {求解方法}
\paragraph{分离变量法}
适用于可分离变量的微分方程。
\begin{itemize}
    \item 例子:\(\frac{dy}{dx} = g(y)h(x)\)
    \[
    \frac{1}{g(y)}dy = h(x)dx \implies \int \frac{1}{g(y)}dy = \int h(x)dx
    \]
\end{itemize}

\paragraph{积分因子法}
适用于一阶线性微分方程。
\begin{itemize}
    \item 例子:\( \frac{dy}{dx} + P(x)y = Q(x) \)
    \[
    \mu(x) = e^{\int P(x)dx}, \quad y \cdot \mu(x) = \int Q(x) \mu(x) dx + C
    \]
\end{itemize}

\paragraph{特征方程法}
适用于常系数线性微分方程。
\begin{itemize}
    \item 例子:\( ay'' + by' + cy = 0 \)
    \item 特征方程:\( ar^2 + br + c = 0 \)
    \item 根据特征方程根的情况,求得通解。
\end{itemize}

\paragraph{级数解法}
适用于复杂方程,通过级数展开求解。
\begin{itemize}
    \item 例子:方程 \( y'' + xy = 0 \) 的级数解法。
\end{itemize}

\subsubsection {例子:人口增长模型}
假设一个地区的人口每年按固定增长率 \( r \) 增长,初始人口为 \( P_0 \)。
\begin{enumerate}
    \item \textbf{确定变量}:设 \( P(t) \) 为第 \( t \) 年的人口。
    \item \textbf{建立微分方程}:
    \[
    \frac{dP}{dt} = rP
    \]
    \item \textbf{初始条件}:\( P(0) = P_0 \)。
    \item \textbf{求解微分方程}:分离变量并积分:
    \[
    \frac{dP}{P} = r dt \implies \ln P = rt + C \implies P = P_0 e^{rt}
    \]
\end{enumerate}

\subsection {非线性微分方程模型的特殊性质}
\subsubsection {基本概念和理论}
\begin{itemize}
    \item \textbf{非线性微分方程}:方程中变量及其导数是非线性的。常见形式为:
    \[
    \frac{d^n y}{dx^n} + f(y, \frac{dy}{dx}, \cdots) = 0
    \]
\end{itemize}

\subsubsection {特殊性质}
\begin{itemize}
    \item \textbf{解的唯一性}:非线性微分方程的解可能不是唯一的。
    \item \textbf{存在性}:解可能在某一区间内存在,但在整个定义域上不存在。
    \item \textbf{稳定性}:解的稳定性分析是非线性微分方程的重要研究内容。
\end{itemize}

\subsubsection {例子:Logistic增长模型}
假设一个地区的人口增长受到环境资源的限制,满足Logistic增长模型。
\begin{enumerate}
    \item \textbf{确定变量}:设 \( P(t) \) 为第 \( t \) 年的人口。
    \item \textbf{建立微分方程}:
    \[
    \frac{dP}{dt} = rP \left(1 - \frac{P}{K}\right)
    \]
    其中,\( r \) 是增长率,\( K \) 是环境容纳量。
    \item \textbf{初始条件}:\( P(0) = P_0 \)。
    \item \textbf{求解微分方程}:分离变量并积分:
    \[
    \frac{dP}{P(1 - \frac{P}{K})} = r dt
    \]
    积分后得到:
    \[
    P(t) = \frac{K P_0 e^{rt}}{K + P_0 (e^{rt} - 1)}
    \]
\end{enumerate}

\subsection {熟悉微分方程的解法}
\subsubsection {解法总结}
\paragraph{分离变量法}
适用于可分离变量的微分方程。
\begin{itemize}
    \item 例子:\(\frac{dy}{dx} = g(y)h(x)\)
    \[
    \frac{1}{g(y)}dy = h(x)dx \implies \int \frac{1}{g(y)}dy = \int h(x)dx
    \]
\end{itemize}

\paragraph{积分因子法}
适用于一阶线性微分方程。
\begin{itemize}
    \item 例子:\( \frac{dy}{dx} + P(x)y = Q(x) \)
    \[
    \mu(x) = e^{\int P(x)dx}, \quad y \cdot \mu(x) = \int Q(x) \mu(x) dx + C
    \]
\end{itemize}

\paragraph{特征方程法}
适用于常系数线性微分方程。
\begin{itemize}
    \item 例子:\( ay'' + by' + cy = 0 \)
    \item 特征方程:\( ar^2 + br + c = 0 \)
    \item 根据特征方程根的情况,求得通解。
\end{itemize}

\paragraph{级数解法}
适用于复杂方程,通过级数展开求解。
\begin{itemize}
    \item 例子:方程 \( y'' + xy = 0 \) 的级数解法。
\end{itemize}

\subsubsection {例子总结}
\paragraph{例子1:一阶线性微分方程}
假设某种药物在人体内的代谢速率与其浓度成正比,初始浓度为 \( C_0 \)。
\begin{enumerate}
    \item \textbf{确定变量}:设 \( C(t) \) 为第 \( t \) 时刻的药物浓度。
    \item \textbf{建立微分方程}:
    \[
    \frac{dC}{dt} = -kC
    \]
    \item \textbf{初始条件}:\( C(0) = C_0 \)。
    \item \textbf{求解微分方程}:分离变量并积分:
    \[
    \frac{dC}{C} = -k dt \implies \ln C = -kt + C_1 \implies C = C_0 e^{-kt}
    \]
\end{enumerate}

\paragraph{例子2:二阶常系数线性微分方程}
假设一个质量为 \( m \) 的物体在弹簧上做简谐运动,弹簧常数为 \( k \)。
\begin{enumerate}
    \item \textbf{确定变量}:设 \( x(t) \) 为第 \( t \) 时刻的位移。
    \item \textbf{建立微分方程}:
    \[
    m \frac{d^2 x}{dt^2} + kx = 0
    \]
    \item \textbf{特征方程}:
    \[
    mr^2 + k = 0 \implies r^2 = -\frac{k}{m} \implies r = \pm i\sqrt{\frac{k}{m}}
    \]
    \item \textbf{求解微分方程}:通解为:
    \[
    x(t) = A \cos \left(\sqrt{\frac{k}{m}} t \right) + B \sin \left(\sqrt{\frac{k}{m}} t \right)
    \]
\end{enumerate}

\newpage

\section {决策论模型}
\subsection {基本理论及其应用}
\subsubsection {基本概念}
\begin{itemize}
    \item \textbf{决策}:在多个可行方案中选择一个方案的过程。
    \item \textbf{决策者}:做出决策的人或组织。
    \item \textbf{方案}:可供选择的行动或策略。
    \item \textbf{状态}:影响决策结果的外部环境或条件。
    \item \textbf{结果}:每个方案在不同状态下的可能结果。
\end{itemize}

\subsubsection {决策问题的类型}
\begin{itemize}
    \item \textbf{确定型决策}:所有信息完全已知,结果确定。
    \item \textbf{风险型决策}:每个状态的概率已知,但结果不确定。
    \item \textbf{不确定型决策}:状态的概率未知,结果不确定。
\end{itemize}

\subsection {决策论的基本方法及应用}
\subsubsection {确定型决策}
\paragraph{线性规划}
用于在确定条件下进行资源分配和优化。
\paragraph{例子}
某公司生产两种产品A和B,每种产品的利润分别为3和5,每种产品的生产时间分别为2小时和4小时。公司每天最多工作40小时。产品A和B每天最多可以生产8件和10件。如何安排生产以使利润最大?
\paragraph{数学模型}
\[
\begin{aligned}
\text{最大化} & \quad z = 3x_1 + 5x_2 \\
\text{约束条件} & \quad 
\begin{cases}
2x_1 + 4x_2 \leq 40 \\
x_1 \leq 8 \\
x_2 \leq 10 \\
x_1, x_2 \geq 0
\end{cases}
\end{aligned}
\]
\paragraph{求解}
使用单纯形法或内点法求解最优解。

\subsubsection {风险型决策}
\paragraph{决策树分析}
通过构建决策树,评估每个方案的期望值。
\paragraph{例子}
某公司计划开发新产品,有两种方案:A和B。方案A成功的概率为0.7,成功时收益为100万元,失败时损失为20万元。方案B成功的概率为0.6,成功时收益为120万元,失败时损失为30万元。如何选择方案?
\paragraph{决策树构建}
\begin{itemize}
    \item 方案A:期望值 \( E(A) = 0.7 \times 100 + 0.3 \times (-20) = 64 \) 万元。
    \item 方案B:期望值 \( E(B) = 0.6 \times 120 + 0.4 \times (-30) = 60 \) 万元。
\end{itemize}
\paragraph{选择方案}
方案A的期望值较高,选择方案A。

\paragraph{效用理论}
通过构造效用函数,将决策者的风险偏好纳入决策。
\paragraph{例子}
某投资者有两种投资方案:高风险高收益和低风险低收益。如何选择方案?
\paragraph{效用函数构建}
\
\begin{itemize}
    \item 设效用函数为 \( U(x) = \sqrt{x} \)。
    \item 高风险方案:收益的期望值为 \( E(U) = 0.5 \times \sqrt{100} + 0.5 \times \sqrt{0} = 5 \)。
    \item 低风险方案:收益的期望值为 \( E(U) = \sqrt{50} = 7.07 \)。
\end{itemize}
\paragraph{选择方案}
低风险方案的期望效用值较高,选择低风险方案。



\subsubsection {不确定型决策}
\paragraph{乐观准则}
选择使得最好情况收益最大的方案。

\paragraph{悲观准则}
选择使得最坏情况收益最大的方案。

\paragraph{折衷准则}
根据某一系数,在乐观和悲观之间取折衷值。

\paragraph{例子}
某公司计划进入新市场,有两种策略:高投入和低投入。在市场需求高时,高投入收益最大,但在市场需求低时,低投入损失最小。如何选择策略?

\paragraph{收益矩阵}
\[
\begin{array}{ccc}
 & \text{市场需求高} & \text{市场需求低} \\
 \text{高投入} & 100 & -50 \\
 \text{低投入} & 60 & 0 \\
\end{array}
\]

\paragraph{乐观准则}
选择使得最好情况收益最大的方案:
\begin{itemize}
    \item 高投入的最好情况收益为 100。
    \item 低投入的最好情况收益为 60。
    \item 选择高投入策略。
\end{itemize}

\paragraph{悲观准则}
选择使得最坏情况收益最大的方案:
\begin{itemize}
    \item 高投入的最坏情况收益为 -50。
    \item 低投入的最坏情况收益为 0。
    \item 选择低投入策略。
\end{itemize}

\paragraph{折衷准则}
根据某一系数 \(\alpha\) 在乐观和悲观之间取折衷值,通常取 \(\alpha = 0.5\):
\begin{itemize}
    \item 高投入的折衷值为 \(0.5 \times 100 + 0.5 \times (-50) = 25\)。
    \item 低投入的折衷值为 \(0.5 \times 60 + 0.5 \times 0 = 30\)。
    \item 选择低投入策略。
\end{itemize}

\subsubsection{层次分析法及其判断矩阵的手算方法}
\paragraph{基本方法}
层次分析法(AHP)用于多准则决策,通过构建判断矩阵进行成对比较,计算各准则的权重。
\paragraph{手算步骤}
1. 构建判断矩阵:根据准则两两比较的结果,构建判断矩阵。
2. 计算权重向量:对判断矩阵进行归一化处理,计算权重向量。
3. 一致性检验:计算一致性比率,判断判断矩阵的一致性。

\paragraph{例子}
考虑三种准则 \(A, B, C\),判断矩阵为:
\[
\begin{array}{c|ccc}
 & A & B & C \\
\hline
A & 1 & 3 & 1/2 \\
B & 1/3 & 1 & 1/4 \\
C & 2 & 4 & 1 \\
\end{array}
\]

1. 归一化判断矩阵:
\[
\begin{array}{c|ccc}
 & A & B & C \\
\hline
A & 1/5 & 3/9 & 1/8 \\
B & 1/15 & 1/9 & 1/32 \\
C & 2/5 & 4/9 & 1/8 \\
\end{array}
\]
2. 计算权重向量:
\[
w_A = \frac{1/5 + 3/9 + 1/8}{3}, \quad w_B = \frac{1/15 + 1/9 + 1/32}{3}, \quad w_C = \frac{2/5 + 4/9 + 1/8}{3}
\]
3. 一致性检验:
计算一致性比率 \(CR\),如果 \(CR < 0.1\),则通过一致性检验。

\paragraph{计算例子}
\begin{enumerate}
    \item 构建判断矩阵:
    \[
    A = \begin{pmatrix}
    1 & 3 & 1/2 \\
    1/3 & 1 & 1/4 \\
    2 & 4 & 1
    \end{pmatrix}
    \]
    \item 归一化判断矩阵:
    \[
    A_{norm} = \begin{pmatrix}
    1/5 & 3/9 & 1/8 \\
    1/15 & 1/9 & 1/32 \\
    2/5 & 4/9 & 1/8
    \end{pmatrix}
    \]
    \item 计算权重向量:
    \[
    \mathbf{w} = \begin{pmatrix}
    (1/5 + 3/9 + 1/8)/3 \\
    (1/15 + 1/9 + 1/32)/3 \\
    (2/5 + 4/9 + 1/8)/3
    \end{pmatrix}
    \]
    \item 计算一致性指标 \(CI\) 和一致性比率 \(CR\):
    \[
    CI = \frac{\lambda_{\max} - n}{n-1}, \quad CR = \frac{CI}{RI}
    \]
    其中,\(\lambda_{\max}\) 是判断矩阵的最大特征值,\(n\) 是判断矩阵的阶数,\(RI\) 是随机一致性指数。

\end{enumerate}

\subsection {实际应用}
\subsubsection {例子1:项目管理}
某项目经理需要在多个项目中进行资源分配,以最大化公司收益。经理可以使用线性规划方法,构建资源分配模型,并求解最优解。
\begin{enumerate}
    \item \textbf{确定变量}:设 \( x_i \) 为分配给项目 \( i \) 的资源量。
    \item \textbf{建立目标函数}:最大化公司收益:
    \[
    z = \sum_{i=1}^{n} c_i x_i
    \]
    \item \textbf{建立约束条件}:资源总量限制和各项目的资源需求:
    \[
    \begin{cases}
    \sum_{i=1}^{n} a_i x_i \leq R \\
    x_i \geq 0
    \end{cases}
    \]
    \item \textbf{求解模型}:使用单纯形法求解最优资源分配方案。
\end{enumerate}

\subsubsection {例子2:投资决策}
某投资公司有多个投资项目,每个项目的收益和风险不同。公司可以使用决策树分析或效用理论,选择最优投资组合。
\begin{enumerate}
    \item \textbf{构建决策树}:分析每个投资项目的收益、风险和概率。
    \item \textbf{计算期望值}:根据决策树,计算每个投资组合的期望收益。
    \item \textbf{选择最优方案}:选择期望收益最高的投资组合。
\end{enumerate}

\subsubsection {例子3:供应链管理}
某制造公司需要在多个供应商中选择合作伙伴,以最小化成本和风险。公司可以使用极大极小准则或最小后悔准则,选择最优供应商。
\begin{enumerate}
    \item \textbf{构建收益矩阵}:分析每个供应商在不同市场条件下的收益和损失。
    \item \textbf{选择准则}:根据极大极小准则或最小后悔准则,选择最优供应商。
    \item \textbf{决策分析}:选择最优供应商合作。
\end{enumerate}

\newpage

\section {图和网络模型}
\subsection {最短路的原理与求解}
\subsubsection {基本概念}
\begin{itemize}
    \item \textbf{图}:由顶点和边组成的结构。图可以是有向图或无向图。
    \item \textbf{路径}:顶点之间的连通序列。
    \item \textbf{路径长度}:路径上边的权重之和。
    \item \textbf{最短路径}:从起点到终点的路径长度最短的路径。
\end{itemize}

\subsubsection {最短路径算法}
\paragraph{Dijkstra算法}
Dijkstra算法用于求解单源最短路径问题,适用于非负权图。
\begin{enumerate}
    \item \textbf{初始化}:
        \begin{itemize}
            \item 设起点为 \( s \),终点为 \( t \)。
            \item 设 \( d(s) = 0 \)(起点到自己的距离为0),对其他所有顶点 \( v \),设 \( d(v) = \infty \)(初始距离为无穷大)。
            \item 设集合 \( S = \{s\} \)(已确定最短路径的顶点集合)。
        \end{itemize}
    \item \textbf{迭代}:
        \begin{itemize}
            \item 从未处理的顶点中选择距离最小的顶点 \( u \),加入集合 \( S \)。
            \item 更新从 \( u \) 到其邻接顶点的距离,对于每个邻接顶点 \( v \),如果 \( d(u) + w(u, v) < d(v) \),则更新 \( d(v) = d(u) + w(u, v) \)。
        \end{itemize}
    \item \textbf{重复}:
        \begin{itemize}
            \item 重复步骤2,直到所有顶点都被处理完毕。
        \end{itemize}
    \item \textbf{结果}:
        \begin{itemize}
            \item 得到起点 \( s \) 到所有顶点的最短路径长度 \( d(v) \)。
        \end{itemize}
\end{enumerate}

\paragraph{Bellman-Ford算法}
Bellman-Ford算法用于求解单源最短路径问题,适用于含负权图。
\begin{enumerate}
    \item \textbf{初始化}:
        \begin{itemize}
            \item 设起点为 \( s \),终点为 \( t \)。
            \item 设 \( d(s) = 0 \)(起点到自己的距离为0),对其他所有顶点 \( v \),设 \( d(v) = \infty \)(初始距离为无穷大)。
        \end{itemize}
    \item \textbf{迭代}:
        \begin{itemize}
            \item 对每条边 \( (u, v) \) 进行松弛操作,如果 \( d(u) + w(u, v) < d(v) \),则更新 \( d(v) = d(u) + w(u, v) \)。
        \end{itemize}
    \item \textbf{重复}:
        \begin{itemize}
            \item 重复步骤2,共进行 \( |V| - 1 \) 次迭代。
        \end{itemize}
    \item \textbf{检测负环}:
        \begin{itemize}
            \item 如果第 \( |V| \) 次迭代还能进行松弛操作,则图中存在负环。
        \end{itemize}
    \item \textbf{结果}:
        \begin{itemize}
            \item 得到起点 \( s \) 到所有顶点的最短路径长度 \( d(v) \)。
        \end{itemize}
\end{enumerate}

\paragraph{例子}

\begin{raggedright}
假设有一个有向图,顶点为 \{A, B, C, D, E\},边及其权重为 \{(A, B, 2), (A, C, 4), (B, C, 1), (B, D, 7), (C, E, 3), (D, E, 1)\},求顶点 A 到其他顶点的最短路径。
\end{raggedright}

\paragraph{Dijkstra算法}
\begin{enumerate}
    \item 初始化:\( d(A) = 0 \),\( d(B) = \infty \),\( d(C) = \infty \),\( d(D) = \infty \),\( d(E) = \infty \)。
    \item 选择 \( A \) 作为起点,处理其邻接顶点 \( B \) 和 \( C \),更新距离:\( d(B) = 2 \),\( d(C) = 4 \)。
    \item 选择距离最小的顶点 \( B \),处理其邻接顶点 \( C \) 和 \( D \),更新距离:\( d(C) = 3 \),\( d(D) = 9 \)。
    \item 选择距离最小的顶点 \( C \),处理其邻接顶点 \( E \),更新距离:\( d(E) = 6 \)。
    \item 选择距离最小的顶点 \( E \),处理其邻接顶点 \( D \),更新距离:\( d(D) = 7 \)。
    \item 选择顶点 \( D \),所有顶点都已处理完毕。
    \item 结果:A 到其他顶点的最短路径长度分别为 \( d(B) = 2 \),\( d(C) = 3 \),\( d(D) = 7 \),\( d(E) = 6 \)。
\end{enumerate}

\subsection{Floyd/Dijkstra求最短路的手算方法}
\subsubsection{Dijkstra算法}
\paragraph{基本方法}
Dijkstra算法用于求解单源最短路径问题,适用于非负权图。
\paragraph{手算步骤}
1. 初始化:设起点为 \(s\),将起点到自己的距离设为0,到其他顶点的距离设为无穷大。
2. 选择顶点:选择一个未处理的顶点,其距离值最小,记为当前顶点。
3. 更新距离:更新当前顶点的所有邻接顶点的距离:
\[
d(v) = \min(d(v), d(u) + w(u, v))
\]
4. 重复步骤2和3,直到所有顶点都被处理。

\paragraph{例子}
考虑如下图:
\[
\begin{array}{c|cccc}
 & A & B & C & D \\
\hline
A & 0 & 1 & 4 & \infty \\
B & 1 & 0 & 2 & 6 \\
C & 4 & 2 & 0 & 3 \\
D & \infty & 6 & 3 & 0 \\
\end{array}
\]
从A到其他顶点的最短路径:
\[
\begin{aligned}
d(A) & = 0 \\
d(B) & = 1 \\
d(C) & = 3 \\
d(D) & = 6
\end{aligned}
\]

\subsubsection{Floyd算法}
\paragraph{基本方法}
Floyd算法用于求解多源最短路径问题,适用于任意权图。
\paragraph{手算步骤}
1. 初始化:构造距离矩阵,将图中所有顶点对的距离初始化。
2. 更新距离:对每一对顶点 \(i, j\),检查是否通过中间顶点 \(k\) 可以缩短距离,如果是则更新距离:
\[
d(i, j) = \min(d(i, j), d(i, k) + d(k, j))
\]
3. 重复步骤2,直到所有顶点对的距离都被更新。

\paragraph{例子}
考虑如下图:
\[
\begin{array}{c|cccc}
 & A & B & C & D \\
\hline
A & 0 & 1 & 4 & \infty \\
B & 1 & 0 & 2 & 6 \\
C & 4 & 2 & 0 & 3 \\
D & \infty & 6 & 3 & 0 \\
\end{array}
\]
最终的距离矩阵:
\[
\begin{array}{c|cccc}
 & A & B & C & D \\
\hline
A & 0 & 1 & 3 & 6 \\
B & 1 & 0 & 2 & 5 \\
C & 3 & 2 & 0 & 3 \\
D & 6 & 5 & 3 & 0 \\
\end{array}
\]


\subsection {欧拉图的概念,判定,求欧拉回路的方法}
\subsubsection {欧拉图的概念}
\begin{itemize}
    \item \textbf{欧拉图}:一个图中包含一个遍历每条边恰好一次的回路(欧拉回路)。
    \item \textbf{欧拉路径}:一个图中包含一个遍历每条边恰好一次的路径。
\end{itemize}

\subsubsection {判定欧拉图的方法}
\begin{itemize}
    \item \textbf{无向图}:所有顶点的度数均为偶数,且图是连通的。
    \item \textbf{有向图}:所有顶点的入度等于出度,且图的每个顶点在强连通分量内。
\end{itemize}

\subsubsection {求欧拉回路的方法}
\paragraph{Hierholzer算法}
\begin{enumerate}
    \item 选择起点:从一个顶点出发。
    \item 构造回路:沿着未访问过的边遍历,直到回到起点,形成一个回路。
    \item 处理剩余边:如果图中还有未访问的边,从当前回路中的某个顶点出发,重复上述过程,直到所有边都被访问。
    \item 合并回路:将所有回路合并成一个欧拉回路。
\end{enumerate}

\paragraph{例子}
考虑如下无向图:
\[
\text{顶点}: \{A, B, C, D\} \\
\text{边}: \{(A, B), (A, D), (B, C), (C, D), (D, B)\}
\]
1. 检查所有顶点度数均为偶数,且图是连通的。
2. 选择起点A,构造回路A-B-D-A。
3. 处理剩余边,从D出发构造回路D-B-C-D。
4. 合并回路得到欧拉回路A-B-D-B-C-D-A。

\subsection{哈密尔顿图的概念,判定,求哈密尔顿回路的方法}
\subsubsection{哈密尔顿图的概念}
\begin{itemize}
    \item \textbf{哈密尔顿图}:一个图中包含一个遍历每个顶点恰好一次的回路(哈密尔顿回路)。
    \item \textbf{哈密尔顿路径}:一个图中包含一个遍历每个顶点恰好一次的路径。
\end{itemize}

\subsubsection{判定哈密尔顿图的方法}
\begin{itemize}
    \item \textbf{Dirac定理}:对于 \( n \) 个顶点的简单图,如果每个顶点的度数至少为 \( n/2 \),则该图是哈密尔顿图。
    \item \textbf{Ore定理}:对于 \( n \) 个顶点的简单图,如果每对不相邻顶点的度数之和至少为 \( n \),则该图是哈密尔顿图。
    \item \textbf{无确定性算法}:哈密尔顿回路的判定和求解没有确定性的多项式时间算法,只能使用搜索方法。
\end{itemize}

\subsubsection{求哈密尔顿回路的方法}
\paragraph{回溯法}
\begin{enumerate}
    \item \textbf{初始选择}:选择一个顶点作为起点。
    \item \textbf{递归构造路径}:从起点出发,依次选择未访问的顶点,构造路径。
    \item \textbf{回溯}:如果当前选择不能构成哈密尔顿回路,则回溯到上一步,选择其他顶点继续搜索。
    \item \textbf{终止条件}:找到一条哈密尔顿回路或搜索完所有可能的路径。
\end{enumerate}

\paragraph{例子}
考虑如下无向图:
\[
\text{顶点}: \{A, B, C, D, E\} \\
\text{边}: \{(A, B), (A, C), (A, E), (B, C), (B, D), (C, D), (C, E), (D, E)\}
\]

\begin{enumerate}
    \item \textbf{初始选择}:选择顶点A作为起点。
    \item \textbf{递归构造路径}:
    \begin{itemize}
        \item 从A出发,选择B,当前路径为A-B。
        \item 从B出发,选择C,当前路径为A-B-C。
        \item 从C出发,选择D,当前路径为A-B-C-D。
        \item 从D出发,选择E,当前路径为A-B-C-D-E。
        \item 检查是否可以从E回到A形成回路。
    \end{itemize}
    \item \textbf{终止条件}:路径A-B-C-D-E-A构成哈密尔顿回路。
\end{enumerate}

\paragraph{手算步骤}
\begin{enumerate}
    \item 从一个顶点开始,尝试每个可能的下一步,记录已访问的顶点。
    \item 如果当前路径包含了所有顶点,并且最后一个顶点可以回到起点,则找到一个哈密尔顿回路。
    \item 如果当前路径不能继续,或没有构成回路,则回溯到上一步,尝试其他可能的路径。
    \item 重复以上步骤,直到找到哈密尔顿回路或搜索完所有可能的路径。
\end{enumerate}


\subsection {最小生成树的原理与求解}
\subsubsection {基本概念}
\begin{itemize}
    \item \textbf{生成树}:连接图中所有顶点且无环的子图。
    \item \textbf{最小生成树}:边的权重和最小的生成树。
\end{itemize}

\subsubsection {最小生成树算法}
\paragraph{Kruskal算法}
\begin{enumerate}
    \item \textbf{初始化}:
        \begin{itemize}
            \item 将图的所有边按权重从小到大排序。
            \item 初始化森林,每个顶点为一个单独的树。
        \end{itemize}
    \item \textbf{迭代}:
        \begin{itemize}
            \item 从小到大选择边,如果边连接的两个顶点在不同的树中,则将这条边加入生成树,并合并这两个树。
        \end{itemize}
    \item \textbf{结果}:
        \begin{itemize}
            \item 最小生成树包含 \( |V| - 1 \) 条边。
        \end{itemize}
\end{enumerate}

\subsubsection {最小生成树的手算方法}
\paragraph{Kruskal算法}
1. 将所有边按权重从小到大排序。
2. 初始化森林,每个顶点为一个单独的树。
3. 依次选择最小的边,如果边连接的两个顶点在不同的树中,则将这条边加入生成树,并合并这两个树。
4. 直到生成树包含 \( |V| - 1 \) 条边。

\paragraph{Prim算法}
1. 从任意一个顶点出发,将其加入生成树。
2. 将该顶点的所有邻接边加入边集。
3. 从边集中选择权重最小的边,如果该边连接的顶点未在生成树中,则将该边和顶点加入生成树,并将新顶点的所有邻接边加入边集。
4. 重复步骤3,直到所有顶点都在生成树中。

\paragraph{例子}
考虑如下无向图:
\[
\begin{array}{c|ccccc}
 & A & B & C & D & E \\
\hline
A & 0 & 2 & 3 & 0 & 0 \\
B & 2 & 0 & 1 & 5 & 0 \\
C & 3 & 1 & 0 & 4 & 0 \\
D & 0 & 5 & 4 & 0 & 2 \\
E & 0 & 0 & 0 & 2 & 0 \\
\end{array}
\]
\paragraph{Kruskal算法}
1. 将所有边按权重排序:\( (B, C, 1) \),\( (A, B, 2) \),\( (D, E, 2) \),\( (A, C, 3) \),\( (C, D, 4) \),\( (B, D, 5) \)。
2. 选择边 \( (B, C, 1) \),加入生成树。
3. 选择边 \( (A, B, 2) \),加入生成树。
4. 选择边 \( (D, E, 2) \),加入生成树。
5. 选择边 \( (A, C, 3) \),加入生成树。
6. 结果:最小生成树为 \( \{(B, C, 1), (A, B, 2), (D, E, 2), (A, C, 3)\} \)。


\paragraph{Prim算法}
\begin{enumerate}
    \item \textbf{初始化}:
        \begin{itemize}
            \item 选择任意一个顶点作为起点,将其加入生成树。
            \item 将该顶点的所有邻接边加入边集。
        \end{itemize}
    \item \textbf{迭代}:
        \begin{itemize}
            \item 从边集中选择权重最小的边,如果该边连接的顶点未在生成树中,则将该边和顶点加入生成树,并将新顶点的所有邻接边加入边集。
        \end{itemize}
    \item \textbf{重复}:
        \begin{itemize}
            \item 重复步骤2,直到所有顶点都在生成树中。
        \end{itemize}
    \item \textbf{结果}:
        \begin{itemize}
            \item 最小生成树包含 \( |V| - 1 \) 条边。
        \end{itemize}
\end{enumerate}

\paragraph{例子}
假设有一个无向图,顶点为 \{A, B, C, D, E\} ,边及其权重为 \{(A, B, 2), (A, C, 3), (B, C, 1), (B, D, 5), (C, E, 4), (D, E, 2)\} ,求最小生成树。

\paragraph{Kruskal算法}
\begin{enumerate}
    \item 初始化:按权重排序边:\( (B, C, 1) \),\( (A, B, 2) \),\( (D, E, 2) \),\( (A, C, 3) \),\( (C, E, 4) \),\( (B, D, 5) \)。
    \item 选择边 \( (B, C, 1) \),加入生成树。
    \item 选择边 \( (A, B, 2) \),加入生成树。
    \item 选择边 \( (D, E, 2) \),加入生成树。
    \item 选择边 \( (A, C, 3) \),加入生成树。
    \item 结果:最小生成树为 \( \{(B, C, 1), (A, B, 2), (D, E, 2), (A, C, 3)\} \)。
\end{enumerate}

\subsection {最大流的原理与求解}
\subsubsection {基本概念}
\begin{itemize}
    \item \textbf{流网络}:一个有向图,每条边有一个容量和一个流量,流量不能超过容量。
    \item \textbf{源点}:流的起点。
    \item \textbf{汇点}:流的终点。
    \item \textbf{流量}:从源点到汇点的总流量。
\end{itemize}

\subsubsection {最大流算法}
\paragraph{Ford-Fulkerson算法}
\begin{enumerate}
    \item \textbf{初始化}:
        \begin{itemize}
            \item 所有边的初始流量为0。
        \end{itemize}
    \item \textbf{寻找增广路径}:
        \begin{itemize}
            \item 使用深度优先搜索或广度优先搜索,在残差网络中寻找从源点到汇点的增广路径。
        \end{itemize}
    \item \textbf{更新流量}:
        \begin{itemize}
            \item 沿增广路径增加流量,更新残差网络。
        \end{itemize}
    \item \textbf{重复}:
        \begin{itemize}
            \item 重复步骤2和步骤3,直到无法找到增广路径。
        \end{itemize}
    \item \textbf{结果}:
        \begin{itemize}
            \item 源点到汇点的最大流量即为最终流量。
        \end{itemize}
\end{enumerate}

\paragraph{Edmonds-Karp算法}
Edmonds-Karp算法是Ford-Fulkerson算法的一种实现,使用广度优先搜索寻找增广路径。

\paragraph{例子}
假设有一个流网络,顶点为 \{S, A, B, C, T\} ,边及其容量为 \{(S, A, 10), (S, C, 10), (A, B, 4), (A, C, 2), (C, B, 8), (B, T, 10), (C, T, 10)\} ,求源点S到汇点T的最大流。

\paragraph{Ford-Fulkerson算法}
\begin{enumerate}
    \item 初始化:所有边的初始流量为0。
    \item 寻找增广路径:使用广度优先搜索找到增广路径 \( S \rightarrow A \rightarrow B \rightarrow T \),容量为4。
    \item 更新流量:沿增广路径增加流量,更新残差网络。
    \item 重复寻找增广路径:找到路径 \( S \rightarrow C \rightarrow T \),容量为10。
    \item 更新流量:沿增广路径增加流量,更新残差网络。
    \item 重复寻找增广路径:找到路径 \( S \rightarrow A \rightarrow C \rightarrow B \rightarrow T \),容量为2。
    \item 更新流量:沿增广路径增加流量,更新残差网络。
    \item 无法找到增广路径,结束。
    \item 结果:最大流量为16。
\end{enumerate}

\end{document}
